\chapter{Erweiterungsmöglichkeiten}

Eine möglich Erweiterung des Frameworks wäre eine Änderung der Schnittstelle. Anstatt wie bisher ein definiertes Interface anzubieten,
könnte man die einzelnen Interface Funktionen auch als Annotationen umsetzen. Eine Klasse im Datenmodell müsste dann die Klasse mit einer
Annotation versehen, die zeigt, dass diese in der Benutzeroberfläche angezeigt werden soll. 
Zusätzlich gäbe es eine Annotation, die über den Assoziationen innerhalb der Klasse steht. Die passenden Setter und Getter Methoden, die in der
ModelList benötigt werden, wären über Reflection zugänglich. Der Typ der Klasse wäre über Reflection ebenfalls einfach zu erhalten.
Diese Erweiterung nimmt dem Nutzer des Frameworks mehr Arbeit ab. Er muss lediglich die Assoziationen und die Klasse mit Annotationen versehen, anstatt
Kenntnis über ein Interface zu haben.\\
\\
Eine weitere Erweiterungsmöglichkeit, ist die automatische Generierung von Templates. Basierend auf dem Quellcode müsste irgendwo hinterlegt sein,
wie ein Datentyp interpretiert werden soll. Beispielsweise ist ein Boolean eine Checkbox, ein Datum ein Date-Element und ein String ein Textfeld.
Diese Erweiterung ist allerdings recht restriktiv gegenüber der Benutzerobefläche, alternativ könnte man dem Nutzer die Möglichkeit geben über spezielle 
Annotationen zu bestimmen auf welche Weise ein Attribut abgebildet werden soll. Alledings ist auch dies mit Restrktionen verbunden - Man gewinnt jedoch an Effizienz, da
nun nur noch das Datenmodell implementiert werden muss.

