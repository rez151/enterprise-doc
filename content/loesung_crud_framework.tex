\chapter{CRUD-Framework}

\section{Beschreibung und Lösungsstrategie}
Die Grundidee hinter dem Framework basiert auf dem Gedanken, dass alle simplen \acs{CRUD}-Anwendungen ein gleiches Verhalten aufweisen, da sie die selbe Funktionalität
zur Verfügung stellen, nämlich \textit{Lesen, Schreiben, Löschen und Ändern}.
Wenn man eine weitestgehend individuelle Benutzeroberfläche außer Acht lässt, liegt der einzige Unterschied zwischen den Anwendungen im Inhalt und der Bedeutung der Datenmodelle.
Darauf basierend braucht es auch an das Datenmodel angepasste Darstellungen in der Benutzeroberfläche.
Das CRUD-Framework geht davon aus, dass ein Datenmodel einer \acs{CRUD}-Anwendung aus drei Komponenttypen besteht. 
Die \enquote{Viewable}-Komponenten, die im Normalfall den Klassen des Datenmodels und den JPA-Entities. 
Die \enquote{ModelLists}, die Sammlungen von Viewable-Komponenten darstellen. 
Die \enquote{Associations}, die eine Erweiterung der ModelList sind. Sie bilden die Beziehungen zwischen den einzelnen Datenmodel-Elementen.\\
\\
Diese abstrakte Darsellung der Datenmodel-Elemente ist ausreichend um die gesamte Funktionalität einer CRUD-Anwendung abbilden zu
können. Diese Eigenschaft macht sich das \acs{CRUD}-Framework zu nutze. Ein Anwender muss nur noch die definierte Schnittstelle in sein Datenmodel implementieren und 
passende XHTML-Temlates für die Darstellung der Elemente erstellen.

\section{Anforderungen}

Das Framework muss in der Lage sein ein Datenmodell und dazugehörige Templates im XTHML Format entgegen zunehmen und eine funktionierende
CRUD-Applikation daraus zu machen. Das Framework soll die gesamte Steuerung der Applikation übernehmen, indem es eine
Reihe von Steuereinheiten und Interfaces zur Verfügung stellt. Einer der Hauptansprüche des Systems, ist eine
automatische Erkennung der Relationen zwischen den Datenmodellen und die richtige Abbildung derer.
Desweiteren sind Hauptanforderungen an das Framework eine Schnittstelle für Fehlermeldungen und Logging. 
Die Daten werden vom System mittels JPA persistiert, sofern das Datenmodell JPA-Entitäten implementiert.

\begin{footnotesize}
	\begin{longtable}[l]{|p{3,0cm}| p{10,5cm} |}
		\caption{Anforderungen CRUD Framework}
		\label{tab:gui_model_desc}
		\hline
		\textbf{FA 1.1} &
		Das System muss die Integration von nahezu beliebigen Datenmodellen ermöglichen.
		\hline
		\textbf{FA 1.2} &
		Das System muss nach der Integration, die Erstellung von neuen Elementen des Datenmodels ermöglichen.
		\hline
		\textbf{FA 1.3} &
		Das System muss die gängigen \acl{CRUD}-Operationen ermöglichen
		\hline
		\textbf{FA 2} & 
		Das System muss in der Lage sein Relationen zwischen den Elementen des Datenmodels darstellen zu können
		\hline
		\textbf{FA 3.1} &
		Die Objektansichten müssen für den Anwendungsfall angepasst agepast werden können
		\hline
		\textbf{FA 3.2} &
		Innerhalb der Templates wird grundsätzlich die Variable \enquote{object} verwendet
		\hline
		\textbf{FA 4} &
		Das System muss in der Lage sein Systemifromationen und Fehlermeldungen in der UI anzuzeigen
		\hline
		\textbf{FA 5} &
		Das System muss die Benutzerauthentifizierung ermöglichen
		\hline
		\textbf{FA 6.1} &
		Das System soll durch den Ansatz des \acs{JPA} die Datenpersistenz sicher stellen
		\hline
		\textbf{FA 6.2} &
		Das System soll einen Austausch der Datenquelle ermöglichen 
		\hline
		 &
		\hline
		\textbf{NFA 1} &
		 Die Antwortzeiten des Systems dürfen nicht länger als 5 Sekunden sein.
		\hline
		\textbf{NFA 2} &
		Das System soll auf allen Systemen lauffähig sein, die einen Internet-Browser besitzen.
		\hline
		\textbf{NFA 3} &
		Die \acs{UI} des Systems soll intuitiv bedienbar sein 
		\hline
		\textbf{NFA 4} &
		Die Veränderung des Datenmodels für den Anwendungsfall soll ohne großen implementierungs Aufwand möglich sein
		\hline
		\textbf{NFA 5} &
		Die Systemarchitecktur soll dokumentiert werden
		\hline
		\end{longtable}
\end{footnotesize}


