\chapter{Referenz-Anwendung}

\section{Anforderungen}

Das Framework muss in der Lage sein ein Datenmodell und dazugehörige Templates im XTHML Format entgegen zunehmen und eine funktionierende
CRUD-Applikation daraus zu machen. Das Framework soll die gesamte Steuerung der Applikation übernehmen, indem es eine
Reihe von Steuereinheiten und Interfaces zur Verfügung stellt. Einer der Hauptansprüche des Systems, ist eine
automatische Erkennung der Relationen zwischen den Datenmodellen und die richtige Abbildung derer.
Desweiteren sind Hauptanforderungen an das Framework eine Schnittstelle für Fehlermeldungen und Logging. 
Die Daten werden vom System mittels JPA persistiert, sofern das Datenmodell JPA-Entitäten implementiert.

%\begin{footnotesize}
%	\begin{longtable}[l]{|p{3,0cm}| p{10,5cm} |}
%		\caption{Klassenbeschreibung}
%		\label{tab:gui_model_desc}
%		\hline
%		\textbf{FA 1.1} &
%		Das System muss die Integration von nahezu beliebigen Datenmodellen ermöglichen.
%		\hline
%		\textbf{FA 1.2} &
%		Das System muss nach der Integration, die Erstellung von neuen Elementen des Datenmodels ermöglichen.
%		\hline
%		\textbf{FA 1.3} &
%		Das System muss die gängigen \acl{CRUD}-Operationen ermöglichen
%		\hline
%		\textbf{FA 2} & 
%		Das System muss in der Lage sein Relationen zwischen den Elementen des Datenmodels darstellen zu können
%		\hline
%		\textbf{FA 3.1} &
%		Die Objektansichten müssen für den Anwendungsfall angepasst werden können
%		\hline
%		\textbf{FA 3.2} &
%		Innerhalb der Templates wird grundsätzlich die Variable \enquote{object} verwendet
%		\hline
%		\textbf{FA 4} &
%		Das System muss in der Lage sein Systeminformationen und Fehlermeldungen in der UI anzuzeigen
%		\hline
%		\textbf{FA 5} &
%		Das System muss die Benutzerauthentifizierung ermöglichen
%		\hline
%		\textbf{FA 6.1} &
%		Das System soll durch den Ansatz des \acs{JPA} die Datenpersistenz sicher stellen
%		\hline
%		\textbf{FA 6.2} &
%		Das System soll einen Austausch der Datenquelle ermöglichen 
%		\hline
%		 &
%		\hline
%		\textbf{NFA 1} &
%		 Die Antwortzeiten des Systems dürfen nicht länger als 5 Sekunden sein.
%		\hline
%		\textbf{NFA 2} &
%		Das System soll auf allen Systemen lauffähig sein, die einen Internet-Browser besitzen.
%		\hline
%		\textbf{NFA 3} &
%		Das \acs{UI} des Systems soll intuitiv bedienbar sein 
%		\hline
%		\textbf{NFA 4} &
%%5		\hline
%		\textbf{NFA 5} &
%		Die Systemarchitecktur soll dokumentiert werden
%		\hline
%		\end{longtable}
%\end{footnotesize}


