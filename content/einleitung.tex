\chapter{Einführung und Ziele}

Im Rahmen des Praktikums der Lehrveranstaltung \enquote{Enterprise Software Architektur} soll eine Webapplikation
entstehen, bei der alle aktuell gängigen Technologien der Java Enterprise Edition eingesetzt werden sollen. 
Dazu gehören zuallererst die Java Server Faces (JSF), ein Framework zur Entwicklung grafischer Oberflächen in Webanwendungen. 
JSF kann durch den Einsatz von Primefaces um neue Funktionen erweitert werden. Primefaces bietet eine Demoseite\footnote{link zur demoseite} mit allen
verfügbaren Komponenten mit implementierung an, wodurch Einsteiger einen schnelleren Einstieg in die Entwicklung finden können.
Zur Persistierung der Daten soll die Java Persistance API (JPA) eingesetzt werden. JPA bietet eine Schnittstelle, um Objekte einfacher
in eine Datenbank zu Übertragen, ohne viel SQL benutzen zu müssen. Alle Entitäten, die in einer Datenbank persistiert werden müssen,
werden mit den entsprechenden Annotations versehen und mit Hilfe eines Entitätenmanagers können Objekte eines solchen Typs
mit einer Zeile Code in die Datenbank geschrieben und herausgelesen werden. Für manche Anwendungsfälle soll eine Zugriffskontrolle stattfinden,
sodass für die zu entwickelnde Applikation ein Login zur Benutzerauthentifizierung nötig ist. 


