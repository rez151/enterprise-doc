\chapter*{Abstract\markboth{Abstract}{}}
\addcontentsline{toc}{chapter}{Abstract}

The complexity of embedded systems in the automotive industry has increased steadily.
The developement of these complex embedded systems is very time-consuming,
therefore companies try to reuse as much of former projects as possible.
To make an efficient reuse possible, companies are in need of globally recognized standards,
whereby the automotive area originated the \ac{AUTOSAR} standard.
\ac{AUTOSAR} specifies the interfaces between features. Everything that is too specific to be standardized is covered within \textit{complex device drivers}.
\\
The scope of the bachelor thesis is a first introduction of the AUTOSAR-compliant tool \textit{EB Tresos Studio}, for the developement of configuration projects,
which allow a partial generation of complex device driver sourcecode.
\\
\\
Die Komplexität von embedded systems in der Automotive Industrie hat im Laufe der Zeit stetig zugenommen.
Die Entwicklung dieser komplexen embedded systems ist sehr zeitaufwändig, daher versucht man so viel wie möglich wiederzuverwenden.
Um eine effiziente Wiederverwendung zu ermöglichen, benötigt es weltweit anerkannte Standards, 
wodurch der Automotive Bereich den \acf{AUTOSAR} Standard ins Leben gerufen hat.
\ac{AUTOSAR} spezifiziert die Schnittstellen zwischen Features, alles was dadurch nicht abgedeckt ist, wird in \textit{Complex Device Drivern} umgesetzt.
\\
Ziel der Bachelorarbeit ist die Einführung des AUTOSAR-konformen Werkzeuges \textit{EB Tresos Studio}, für die Entwicklung von Konfigurationsprojekten, die 
eine Quellcode Generierung für Teile von Complex Device Drivern ermöglichen soll.
